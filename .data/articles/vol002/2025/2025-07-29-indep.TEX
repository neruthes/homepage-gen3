\iarticle{2025-07-29}{The Mirage of Independent Blog}

Having a WordPress website for the first time was an exciting moment which happened around 2012.
Owning an independent website was nice, and is still nice today,
because it means full customization at will as long as my frontend programming skills permit.

Due to the shortage of payment instruments, it was hard to buy normal domains and normal servers,
and I had to work with limited hosting options.

Later I moved to static site generator because of the rise of GitHub Pages.
It was a time when GitHub Actions was not in place and I had to put everything servable into Git tracking.

In subsequent years, more options appeared and more risks appeared.
Today, CDN is a security necessity and disclosing the real IP address of a small VPS can be harmful.
Domains prices can go really high, and CDN businesses can reject serving websites that they do not like.

Making a domain name canonical and paramount (to represent a person) can bring more harm than good.
This is especially prominent after COVID, because death may catch me anytime soon.
Having a blog carries the need for being interpreted and being remembered;
in this sense, a CMS app running on the web is not the ideal approach, and static site generators only helped slightly.

The traditional idea of having an independent blog has become a mirage, a delusion, and a hoax.
It was a mix of several different pursuits, including to \textbf{make announcements}, to \textbf{welcome guests}, and to \textbf{be remembered}.

The ``to make announcements'' pursuit can be well satisfied by modern public services including Mastodon and Substack;
even GitHub can make a nice option.

The ``to welcome guests'' pursuit should translate to a digital garden \footnote{Jacky Zhao. Digital Gardening. 2021-10-31. {https://jzhao.xyz/posts/digital-gardening}},
where a timeline of articles may only be a small component inside it.
This should be comparable with Serenitea Pot and Animal Crossing,
where people visit gardens and homes of each other.
I remember the good old days with Tencent Qzone where widgets and decorations overweighed content.
The current content-focus trend is neglecting this aspect of psychological need for safe space.

My current approach is super optimized for ``to be remembered''.
I have a shell-based static site generator, a PDF-oriented blog aggregation workflow,
a multi-hostname deployment workflow, and a set of ready-to-deploy archives.
I have made my website super easy to be rebuilt and deployed by other people after my death,
in order to make my central recollections (personal namespace in contrast to various projects)
as accessible as possible for future historians.
Near-death experiences since COVID contributed a lot to this aspect of consideration
as part of the fight against anxiety and depression.

Will I make my digital garden for meeting guests? Probably.

