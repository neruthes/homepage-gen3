\iarticle{2025-07-25}{Recap for Migrating to GNU/Linux Desktop}

Historically I used Windows 98, Windows 2000, and Windows Xp,
before I was initially equipped officially with a computer,
a 13-inch MacBook Pro (2010) which ran Mac OS X Snow Leopard.
I would mark it as my first contact to the UNIX-like world of desktops.

I was Skeuomorphist, and I am still one today.
Snow Leopard was followed by the Flat jihadists who brought a total decimation of the good old days in my 18-ish age.
Steve Jobs was gone, and Apple betrayed my faith in it by murdering the holy reign of Skeuomorphism.

After Mountain Lion, Mavericks was on the verge of the farewell.
Over time, I got Yosemite, El Capitan, etc, and I became CLI supremacist in great lack of faith in having good GUI ever again.
I started looking for alternatives to Mac over the years and only in 2020 I noticed a desirable hardware from Xiaomi.
And for the OS, I had no other choice than GNU/Linux.

Installing Ubuntu 20.04 LTS was easy, but I wanted something else.

I tried AOSC OS, a community-driven distro with an apt/dpkg system and various installation options, and soon fled because it was also a systemd-only distro.
Before I moved away, I worked with configuration options and learned a lot stuff like bootloader, kernel, FS, LUKS, X, window manager.
I got xinit, startx, and i3wm to work together so I would not rely on any jumbo desktop environment; I still use the combination today.
It was a nice learning experience which enabled me to the next step.

In search for non-systemd options, I noticed Gentoo for supporting OpenRC.
Years ago, I had heard of its name and it had been reportedly hard to install and hard to update.
I switched to Gentoo and the installation process was not a challenge for me because I had gathered enough knowledge to maneuver through
cfdisk, cryptsetup, mkfs, mount, untar stage3, fstab, kernel, initramfs, firmware, kernel cmdline, grub-install, grub-mkconfig, kernel drivers,
glibc, USE flags, dhcpcd, wpa\_supplicant, etc.

I have got this far...

\begin{tabu}{XX[2]}
	\toprule
	Component  & Choice                      \\
	\midrule
	Kernel     & 6.15.7 (with Gentoo flavor) \\
	Libc       & Glibc 2.41-r4               \\
	Bootloader & GRUB                        \\
	GUI        & Xorg + i3wm + picom         \\
	Filesystem & Ext4 over LUKS              \\
	\bottomrule
\end{tabu}

