\iarticle{2023-01-30}{The Governmental Lifestyle}

\begin{multicols*}{2}
	\isection{Discussions on Bisexuality}
	In recent years, I occasionally encountered the discussions on how bisexual people should be considered
	in terms of potential spouse-prospects and potential spouses.
	One common concern was that a bisexual person might be at higher risk in
	surrendering to paternal and maternal pressures for heterosexual marriage,
    which would lead to the ultimate betrayal of love and the ultimate destruction of relationship.

	I argued otherwise.
	Suppose that B and D are similar persons where B claims to be bisexual while D remains silent with the presumption of being homosexual.
	The statement from B makes no evidence that B is more likely than D to be non-homosexual;
	D may be concealing the fact that he is also bisexual.
	While it is an acceptable general speculation that bisexual people are more likely to claim to be bisexual,
	in this particular case, the single difference in self-claimed bisexuality does not make B more probably being bisexual than D.

	Later, I realized that this way of thinking was not merely probability theory; it also hinted the idea of Due Process.
	The voluntary disclosure of information of a person from himself shall not be weaponized against himself.

    \isection{Story of Attitudes}
	Recently, I exchanged ideas with a friend who was impressed by seeing his father socially interacting with someone
    in a way so different from what would be used with him and his mother in a family context.
    Being a 27-yo man means that this phenomenon would not be a surprise for him,
    given the knowledge on human society and human minds, but still an impressive scene to witness, in his opinion.

    I enjoyed the story and suggested that many humans had developed the skill of using different attitudes,
    or even different personality customs, in social interactions with different kinds of people,
    which, as a skill was valuable, but I would prefer not to practice.

    Personality flexibility and attitude flexibility are valuable skills, but they come with a burden ---
    the user is compelled to do the computation of searching the optimal attitude
    for each particular person in each particular context of socialization.
    On the contrary, I prefer using the same attitude to everyone in every context.
    It is an energy-saving measure, and it has profound implications ---
    one who cannot accept the truth, of my being what kind of person I am, may not be entitled to surplus attention from me,
    which is an important foundation to facilitate any attitude customization.
    This forms a paradox --- one will have access to my attitude customization operation only if it does not need it.

    I display one face to all people from all backgrounds in all times,
    and I am satisfied with my consistency.

    \isection{Governemt \& Principles}
    A qualified government abides to principles; abiding to (commonly endorsed) principles is the source of governmental legitimacy.
    Abiding to principles frees a government from trivializations of operations and contradictions of decisions.

    I find it a good lifestyle to live like a government.
    In the story of bisexuality, I prioritize the principle of Due Process over exploring suspicions on a per-person basis.
    In the story of attitudes, I prioritize the principle of Consistency over customizing for individual expectations from other people.

    The governmental lifestyle implies that one shall be resistent from foreign influences ---
    any input must be filtered, investigated, and validated, before being submitted to influence top-level policy decisions.
    % This implication might perhaps fit in the ``introvert thinking'' category from Carl Jung;
    % I am not sure how professionally would the people who do not have Ti as the Dominant or Auxiliary function be enjoying this lifestyle.
    
    % Mask of Solitude Basalt
\end{multicols*}

