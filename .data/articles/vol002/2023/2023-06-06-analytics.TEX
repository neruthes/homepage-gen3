\iarticle{2023-06-06}{Basic Model for Analytics Systems}

% \isection{Introduction}

This article introduces a basic model for designing analytics systems on an abstract level.


\isection{Window and Module}

We often inspect the distribution of data points over time.
For example, order creation and user login are common data points.
When some event happens, it leaves a data point in the sight of analytics.

When we ask how many orders were created in a specific day,
we consider the day as a module which contains data points.
When we ask how the order-per-day quantity were fluctuating over a specific month,
we consider the month as a window which contains modules.


\isection{Sensor}

A sensor is a way to collect a data point out of a raw business event.
For example, effective order quantity and sales income are 2 different sensors.
One collects static number 1 out of each effective order while the other collects the `order.totalAmount' information out of it.


\isection{Stem}

When we ask how many orders were created in a specific day,
we are combining the module `day 2023-01-01' and the sensor `quantity of orders'.
This combination is called a stem.

A window is spanned by a series of stems.


\isection{Conclusion}

Now we have defined the collection of data and the presentation of summarized information.
Issues remain for scenarios like state change, and they should be left for individual implementations.

