\iarticle{2023-02-13}{Apologos for Modern Nonreaders of Old Journalism}

\begin{multicols*}{2}
    \isection{Introduction}

	Over the last decade, or the last 2 decades, the influence of old journalism has been in a robust decline.
	Along with the rise of contemporary social network and the proliferation of cellular network services,
	paper-based old journalism, primarily represented by newspapers and monthly/biweekly journals,
	has been in constant withering.

	Accusations often go to the general public.
	The critics coming from self-claimed cultural elites often put the majority, if not the whole, blame on the general public ---
	modern people are not as good in reading serious articles as good-old-days people were.

    Although I am enthusiast in \LaTeX{} and the related toolkit for typesetting and publication,
    I do acknowledge the necessity to write my apologos for the nonreaders of old journalism,
    a big portion, which includes myself, of modern population.

    \isection{Old Curse on Editorship}

    With or without a real genealogical relation with pre-WW1 bourgeoisie,
    Editorship has displayed a feature, which may be rhetorically called a curse ---
    the editors, with the support from the journalists,
    labeled themselves as the producers and curators of information,
    a role so distinguished from the readers,
    who were considered as mere consumers of information and were unable to collect, curate, or digest information.

    A simple categorization of mainstream old journalism (excluding academic journals) may include a few major fields:
    politics (e.g. National Interest),
    creative writing (e.g. Scientific American),
    and fashion (e.g. Vogue).

    In politics, with the aid of whistles, partisan choice has been quite prevailing.
    People react to news events on the basis of partisanship.
    For example, when the case WI vs Kyle Rittenhouse (20 CF 983) gets hot,
    it is easy to expect republican reactions with a pro-gun tone and democrat reactions with a BLM tone.
    In such news contexts, the consumers of information prefer narratives
    which are capable of reinforcing their own political positions,
    and perfectly-neutral reporting is rarely required --- and rarely supplied, too.

    In creative writing, since the introduction of blogging,
    creative writing is no longer a subsection of old newspapers and journals.
    Everyone is entitled to write and to publish.
    With the rise of social media, since news feed was introduced by Facebook,
    the distribution and discovery of products and authors in creative writing have become profoundly easy,
    at least in terms of quantity.
    The increase of ISP bandwidth brought multimedia into creative writing and ``creative content creating''
    may be a better term for the new era.
    Maturing recommendation algorithms have demonstrated great productivity
    and editors can hardly compete in selecting good contents for content consumers.

    In fashion, I would expect similar effects although I am not a regular consumer of fashion news contents.
    Without a decent salary, which is critical to a decent taste,
    an editor can hardly offer any useful insight for the target consumers (i.e. middle class fashion enthusiasts)
    and can only be a mouthpiece for fashion brand marketing teams.
    After decades of lifestyle education, middle class fashion enthusiasts themselves
    may have been developing preferences for tailored stuff,
    rendering traditional ready-to-wear clothing and other ready-to-use stuff less appealing.
    They no longer need some journal editor to tell them what tea/coffee/crystal/watch are good.
    This applies to PC enthusiasts, too.
    They are well aware of AMD/Intel CPUs and AMD/Nvidia video cards;
    Kioxia/Samsung/SanDisk/Zhitai NVMe hard drives are easily compared.
    Intelligences are exchanged in communities every day;
    video creators publish their own benchmarking videos on YouTube and Bilibili.

    The editors used to be, or pretended to be,
    elites in business industries and frontier information.
    Their socioeconomic elite role is no longer maintainable.
    Their information is no longer more accurate or more insightful than that of their target audience;
    not even faster.

    \isection{Cultural Awareness}

    The middle class understand that they do not need some journal editor to educate them
    what tea/coffee/crystal/watch are good.
    The proletariat understand that they do not need to beg the rich to spare a little piece of culture.
    Some ``elite journals'' suck celebrities too shamelessly and practice the entrepreneur worship so disgustingly.
    A journalist is always free to do interviews with Tim Cook or Ben Horowitz,
    but what outcome can a reader expect for reading a journal full of such flatterers?
    For entertainment, Mihoyo and Netflix are more professional;
    for money, Coursera and Udemy offer a lot of STEM knowledge to learn;
    for a more sophisticated understanding of the human society, anthropology and sociology are good subjects to read about.

    The era of social media has been establishing and reinforcing the cultural awareness of everyone.
    Modern people are more educated to exercise their own cultural preferences in content consuming,
    and are less interested in worshiping the irrelevant narcissist editors.
\end{multicols*}

