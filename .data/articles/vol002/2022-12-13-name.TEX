\iarticle{2022-12-13}{Statement on Name Change}

\begin{multicols*}{2}
6 years ago, on 2016-03-10, I wrote \textit{From Joy Neop to Neruthes} in my blog,
the article which indicated the beginning of a new chapter of my sojourn in this world.
Over the last 6 years, I gradually changed the way of indicating my name.
Initially, I changed to `Joy Neop (a.k.a. Neruthes)'; later, `Neruthes (a.k.a. Joy Neop)';
finally, only `Neruthes' per se.
From site to site, from piece to piece;
the process of replacing was like a stream of raindrops.
I have offered everyone a soft and subtle process for getting familiar with this name.

However, it appeared that the soft and subtle process only succeeded partially.
Some old friends did not follow the process as fluently as I hoped.

Also, I might have been known to certain people as Mr Kim in the context of People's Republic of China.
While at this moment I find no ground to renounce Chinese nationality for lacking a nationality (or an ongoing naturalization process)
of another internationally recognized sovereign state,
this way of addressing may only be exceptionally approved as a disguise in front of a person who:
(1) is not a friend or an acquaintance of mine,
(2) is a citizen of People's Republic of China,
(3) has household registration in Innerland,
(4) has no permanent resident status in Hong Kong SAR or Macau SAR, and
(5) is not an active resident in Republic of China.
This exception will be abolished when I acquire nationality of another internationally recognized sovereign state.

By the end of year 2022, I find it a good opportunity to clarify on this subject.
I expect everyone to address me by my canonical name `Neruthes',
and to refrain from using the deprecated ways of addressing which were historically practiced.
\end{multicols*}
